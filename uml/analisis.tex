\chapter{Análisis}

\section{Introducción}

\newpage

\section{Diagrama de Casos de Uso}



\begin{table}[]
\centering
\caption{My caption}
\label{my-label}
\resizebox*{\textwidth}{!}{
\begin{tabular}{|
>{\columncolor[HTML]{B6ECF7}}c |c|}
\hline
\multicolumn{1}{|l|}{\cellcolor[HTML]{B6ECF7}\textbf{}} & \multicolumn{1}{r|}{ID:1}                                                                                       \\ \hline
\textbf{Nombre}                                         & Riego del cultivo                                                                                               \\ \hline
\textbf{Objetivo}                                       & \begin{tabular}[c]{@{}c@{}}Controlar y medir la cantidad de agua suministrada a las plantas.\end{tabular}    \\ \hline
\textbf{Actores}                                        & Cultivador                                                                                                      \\ \hline
\textbf{Escenario Primario}                             & Se hace el riego de manera exitosa.                                                                             \\ \hline
\textbf{Escenario Secundario}                           & \begin{tabular}[c]{@{}c@{}}El riego no se lleva a cabo de la manera prevista (Exceso o déficit)\end{tabular} \\ \hline
\textbf{Escenario Excepcional}                          & \begin{tabular}[c]{@{}c@{}}No se dispone de agua. El sistema de riego no funciona. Lluvias\end{tabular}      \\ \hline
\end{tabular}}
\end{table}

\begin{table}[]
\centering
\caption{Funciona coño}
\label{my-label}
\resizebox*{\textwidth}{!}{
\begin{tabular}[c]{|
>{\columncolor[HTML]{B6ECF7}}c |c|}
\hline
\multicolumn{1}{|l|}{\cellcolor[HTML]{B6ECF7}} & \multicolumn{1}{r|}{ID:2}                                                                                                                                            \\ \hline
\textbf{Nombre}                                & Iluminación del cultivo                                                                                                                                              \\ \hline
\textbf{Objetivo}                              & \begin{tabular}[c]{@{}c@{}}Controlar la cantidad e intensidad de luminosidad que recibe el cultivo\end{tabular}                                                   \\ \hline
\textbf{Actores}                               & Cultivador                                                                                                                                                           \\ \hline
\textbf{Escenario Primario}                    & \begin{tabular}[c]{@{}c@{}}El cultivo recibe la cantidad de luz necesaria que garantice su crecimiento óptimo\end{tabular}                                        \\ \hline
\textbf{Escenarios Secundarios}                & La luz solar se incrementa a causa de un verano y sequia                                                                                                             \\ \hline
\textbf{Escenarios Excepcional}                & \begin{tabular}[c]{@{}c@{}}El sistema que brinda luz y sombra al cultivo deja de funcionar, los eclipses de bogotá que oscurecen salvajemente :'v\end{tabular} \\ \hline
\end{tabular}}
\end{table}

\begin{table}[]
	\centering
	\caption{Este desgraciado se salto a otra seccion :D}
	\label{my-label}
	\resizebox*{\textwidth}{!}{
	\begin{tabular}{|
			>{\columncolor[HTML]{B6ECF7}}c |c|}
		\hline
		\multicolumn{1}{|l|}{\cellcolor[HTML]{B6ECF7}} & \multicolumn{1}{r|}{ID:3}                                                                                                                                                                                            \\ \hline
		\textbf{Nombre}                                & Estado de la tierra del cultivo                                                                                                                                                                                      \\ \hline
		\textbf{Objetivo}                              & Controlar los niveles de ph en la tierra utilizada para los cultivos                                                                                                                                                 \\ \hline
		\textbf{Actores}                               & Cultivador                                                                                                                                                                                                           \\ \hline
		\textbf{Escenario Primario}                    & El cultivo tiene el ph adecuado para el óptimo crecimiento de la planta dependiendo de su tipo                                                                                                                       \\ \hline
		\textbf{Escenarios Secundarios}                & NO ENCONTRE ESCENARIOS SECUNDARIOS :D                                                                                                                                                                                \\ \hline
		\textbf{Escenarios Excepcional}                & El nivel del ph del suelo es bajo a causa de la mala fertilización de los cultivos, lluvias ácidas, entre otros. El nivel de ph del suelo es alto causado por falta de agua en la tierra o mal manejo de los riegos. \\ \hline
	\end{tabular}}
\end{table}

\section{Interacciones}

\newpage

\subsection{Diagrama de Secuencia}

\newpage

\subsection{Diagrama de Comunicación}

\newpage

\subsection{Diagrama de Temporización}

\newpage

\section{Diagramas de Actividades}

\newpage

\section{Diagramas de Actividades}

\newpage

\section{Diagramas de Workflow}

\newpage

\section{Diagramas de Descripción de la Interacción}

\newpage